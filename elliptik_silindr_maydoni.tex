\documentclass[landscape]{scrartcl}
\usepackage{cmap}
\usepackage{amsmath}
\usepackage{geometry}
\usepackage{fancyhdr}
\geometry{
		top=3cm,
		bottom=2cm,
		right=2cm,
		left=3cm
		}
\begin{document}
	\begin{center}
		\Large\textbf{Elliptik silindrning elektr maydoni}
		
		\textit{Yomg`irov Olim}\\
		Mirzo Ulug`bek nomidagi O`zbekiston Milliy Universiteti\\
		Fizika fakulteti\\
		\texttt olimyomgirov1284@gmail.com
	\end{center}
	\pagestyle{empty}
	\pagestyle{fancy}
	
	
	
	
	
	Cheksiz uzun silindr sirti bo`ylab zaryadlar bir jinsli taqsimlangan bo`lsin. Zaryadlarning sirt zichligi $\rho$. Shu bir jinsli cheksiz uzun silindr sirtidagi elektr maydonini hisoblaymiz.
	Ellipsning x va z o`qlari bo`yicha yarim o`qlari a va b ga teng deylik. U holda, zaryadlar hosil qiladigan potensial quyidagi ko`rinishda bo`ladi:
	\begin{equation}
	\varphi(x,z)=
	\rho\int\limits_{-a}^{a}dx'
	\int\limits_{-b\sqrt{1-\frac{x'^{2}}{a^{2}}}}^{b\sqrt{1-\frac{x'^{2}}{a^{2}}}}dz'\ln\frac{1}{(x-x')^{2}+(z-z')^{2}}
	\label{1}
	\end{equation}
	Endi esa, zaryadlar silindrdan tashqarida hosil qilgan maydonning z o`qi bo`yicha tashkil etuvchisini ko`rib chiqamiz:
	\begin{equation}
	\frac{x^{2}}{a^{2}}+\frac{z^{2}}{b^{2}}=1+\delta \ \ \ \ \ (0<\delta\ll 1)
	\label{2}
	\end{equation}
	Uni quyidagicha ifodalashimiz mumkin:
	\begin{equation}
	E_{1}(x,z)\equiv-\dfrac{\partial\varphi}{\partial z}=\rho\int\limits_{-a}^{a}dx'
	\int\limits_{-b\sqrt{1-\frac{x'^{2}}{a^{2}}}}^{b\sqrt{1-\frac{x'^{2}}{a^{2}}}}
	dz'\dfrac{\partial}{\partial z'}
	\ln{\frac{1}{(x-x')^{2}+(z-z')^{2}}}=
	\rho\int\limits_{-a}^{a}dx\prime\ln{\frac{(x-x')^{2}+
			\left(z+b\sqrt{1-\frac{x'^{2}}{a^{2}}}\right)^{2}}{(x-x')^{2}+
			\left(z-b\sqrt{1-\frac{x'^{2}}{a^{2}}}\right)^{2}}}
		\label{3}
	\end{equation}
	Bu integralni hisoblash uchun yangi o`zgaruvchilar kiritamiz:
	\begin{equation}
	x=a\cos\psi,\ \ \ \ \\ x'=a\cos\psi',\ \ \ \ b/a=\varepsilon,\ \ \ \ \\delta+\sin^{2}\psi=\sin^{2}\varphi
	\label{4} 
	\end{equation}
	U holda $\sqrt{1-\frac{x'^{2}}{a^{2}}}=\sin\psi',\ z=b\sin\varphi$ bo`ladi hamda
	\begin{equation}
	E_{z}=a\rho\int\limits_{0}^{\pi}
	\sin\psi' d\psi'
	\ln\frac{(\cos\psi'-\cos\psi)^{2}+\varepsilon^{2}(\sin\psi'+\sin\varphi)^{2}}
	{(\cos\psi'-\cos\psi)^{2}+\varepsilon^{2}(\sin\psi'-\sin\varphi)^{2}}
	=a\rho
	\int\limits_{0}^{\pi}
	\sin\psi' d\psi'
	\ln 
	\frac{\sin^{2}\frac{\psi'+\psi}{2}\sin^{2}\frac{\psi'-\psi}{2}+
		\varepsilon^{2}\sin^{2}\frac{\psi'+\varphi}{2}\cos^{2}
		\frac{\psi'-\varphi}{2}}
	{\sin^{2}\frac{\psi'+\psi}{2}\sin^{2}\frac{\psi'-\psi}{2}+
		\varepsilon^{2}\sin^{2}\frac{\psi'+\varphi}{2}\sin^{2}
		\frac{\psi'-\varphi}{2}}
	\label{5}
	\end{equation}
	$\delta\rightarrow 0$ bo`lganda, $\varphi\rightarrow\psi$. Bu o`tishning uzluksizligidan foydalanib integral ostidagi ifodani quyidagicha yozib olishimiz mumkin:
	\begin{eqnarray}
	E_{z}=a\rho\int\limits_{0}^{\pi}
	\sin\psi' d\psi'
	\ln
	\frac{\sin^{2}\frac{\psi+\psi'}{2}\left[
		\sin^{2}\frac{\psi-\psi'}{2}+
		\varepsilon^{2}\cos^{2}\frac{\psi-\psi'}{2}
		\right]}
	{\sin^{2}\frac{\psi-\psi'}{2}
	\left[
	\sin^{2}\frac{\psi+\psi'}{2}+
	\varepsilon^{2}\cos^{2}\frac{\psi+\psi'}{2}
	\right]}
=a\rho\int\limits_{0}^{\pi}
\sin\varphi d\varphi
\left[
\frac{1-\cos(\varphi+\psi)}
	 {1-\cos(\varphi-\psi)}
	 \frac{1+\frac{\varepsilon^{2}-1}
	 {\varepsilon^{2}+1}\cos(\varphi-\psi)}
 {1+\frac{\varepsilon^{2}-1}
 	{\varepsilon^{2}+1}\cos(\varphi+\psi)}	 
\right]= \\ 
=a\rho\int\limits_{\pi+\varphi}^{\pi-\varphi}
\sin(\varphi+\psi)d\varphi\ln
\frac{1+\frac{\varepsilon^{2}-1}
	{\varepsilon^{2}+1}\cos\varphi}
{1-\cos\psi}
+a\rho\int\limits_{\psi}^{\pi+\psi}
\sin(\varphi-\psi)d\varphi\ln
\frac{1-\cos\psi}
{1+\frac{\varepsilon^{2}-1}
	{\varepsilon^{2}+1}\cos\varphi}
\label{6}
	\end{eqnarray}
	
Birinchi integralda integrallash o`zgaruvchisini almashtiramiz: $\varphi\prime=\pi-\varphi$. Natijada 
\begin{equation}
E_{z}=a\rho\int\limits_{\psi}^{\pi+\psi}
\sin(\varphi-\psi)d\varphi
\left[
\ln
\frac{1+\frac{\varepsilon^{2}-1}
	{\varepsilon^{2}+1}\cos\varphi}
{1-\frac{\varepsilon^{2}-1}
	{\varepsilon^{2}+1}\cos\varphi}
+\ln
\frac{1-\cos\varphi}
	 {1+\cos\varphi}
\right]
=a\rho\left[\Phi
\left(
\psi,\frac{\varepsilon^{2}-1}
		  {\varepsilon^{2}+1} 
\right)
+\Phi(\psi,1)
\right]
\label{7}
\end{equation}	
	bu yerda
	\begin{equation}
	\Phi(\psi,\alpha)=
	\int\limits_{\psi}^{\pi+\psi}
	\sin(\varphi-\psi)d\varphi\ln
	\frac{1-\alpha\cos\varphi}
		 {1+\alpha\cos\varphi}
	\label{8}
	\end{equation}
	\eqref{8}-integralni bo`laklab integrallash mumkin:
	\begin{equation}
	\Phi(\psi,\alpha)=
	-\cos(\varphi-\psi)\ln
		\frac{1-\alpha\cos\varphi}
	{1+\alpha\cos\varphi}
	\Big |_{\psi}^{\pi+\psi}
	+2\pi\int\limits_{\psi}^{\pi+\psi}
	\frac{\cos(\varphi+\psi)\sin\varphi}
		 {1-\alpha^{2}\cos^{2}\varphi}
		 +2\alpha\sin\psi\int\limits_{\psi}^{\pi+\psi}
		 \frac{\sin^{2}\varphi d\varphi}
		 	  {1-\alpha^{2}\cos^{2}\varphi}
		 	  \label{9}
	\end{equation}

	Ko`rinib turibdiki, \eqref{9}-integraldagi birinchi had nolga teng. Ikkinchi hadda integral ostidagi funksiyaning $(0,\psi)$ va $(\pi,\pi+\psi)$
	 intervallardagi qiymati bir-biriga mos keladi. Natijada, $\Phi(\psi,\alpha)$ ni quyidagi ko`rinishda yozishimiz mumkin:
	 \begin{equation}
	 \Phi(\psi,\alpha)=2\alpha\sin\psi\int\limits_{0}^{\pi}\frac{\sin^{2}\varphi d\varphi}{1-\alpha^{2}\cos^{2}\varphi}
	 \label{10}
	 \end{equation}
	$\alpha=1$ hol uchun integral oson hisoblanadi va $2\pi\sin\psi$ ga  teng. $\alpha\ne1$ uchun esa
	\begin{equation}
	\Phi(\psi,\alpha)=\frac{2\pi\sin\psi}{\alpha}
	\left[
	1-(1-\alpha^{2})
	\int\limits_{0}^{\pi}\dfrac{d\varphi}{1-\alpha^{2}\cos^{2}\varphi}
	\right]
	\label{11}
	\end{equation}
	yoki
	\begin{equation}
	\Phi(\psi,\alpha)=\frac{2\pi\sin\psi}{\alpha}
	\left(
	1-\sqrt{1-\alpha^{2}}
	\right)
	\label{12}
	\end{equation}
	Shundan so`ng, \eqref{12} ni \eqref{7}-ga qo`yamiz va \eqref{4}-ni hisobga olgan holda, elliptik silindr maydoni uchun quyidagi ifodan yozamiz:
	\begin{eqnarray}
	E_{x}(x,z)=4\pi\rho\dfrac{a}{b+a}z
	\\
	E_{z}(x,z)=4\pi\rho\dfrac{b}{b+a}x
	\label{13,14}
	\end{eqnarray}
	
	
\end{document}